\documentclass[%
	11pt,
	a4paper,
	utf8,
	%twocolumn
		]{article}	

\usepackage{style_packages/podvoyskiy_article_extended}


\begin{document}
\title{Заметки по большим данным и облачным технологиям}

\author{\itshape Подвойский А.О.}

\date{}
\maketitle

\thispagestyle{fancy}

Здесь приводятся заметки по некоторым вопросам, касающимся больших данных, облачных технологий, машинного обучения, анализа данных, программирования на языках \texttt{Python}, \texttt{R} и прочим сопряженным вопросам так или иначе, затрагивающим работу с данными.


%\shorttableofcontents{Краткое содержание}{1}

\tableofcontents

\section{Основные термины и определения}

\noindent\emph{Витрина данных} (Data Mart) -- срез хранилища данных, представляющий собой массив тематической, узконаправленной информации, ориентированный, например, на пользователей одной рабочей группы или департамента.


\section{Логическая витрина для доступа к большим данным}

Пример. Рассмотрим некоторый промышленный комплекс, обладающий огромным количеством оборудования, обвешанного различными датчиками, регулярно сообщающими сведения о состоянии этого оборудования. Для простоты рассмострим только два аргрегата (котел и резервуар), и три датчика (температуры котла и резервуара, а также давления в котле).

Эти датчики контролируются АСУ разных производителей и выдают информацию в разные хранилища: сведения о температуре и давлении в котле поступают в HBase, а данные о температуре в резервуаре пишутся в лог-файлы, расположенные в HDFS.

Данные о датчиках могут храниться, например, в \texttt{PostgreSQL}, а показания этих датчиков -- в HDFS, HBase и т.п. Теперь пусть мы хотим предоставить аналитику возможность делать запросы. Заранее построить и запрограммировать сложные запросы не получится. Выполнение любого сложного, тяжелого запроса требует связывания данных из разных источников, в том числе из находящихся за пределами нашего модельного примера. Извне могут поступать, например, справочные сведения о рабочих диапазонах температуры и давления для разных видов оборудования, фасетные классификаторы, позволяющие определить, какое оборудование является маслонаполненным и др. Все подобные запросы аналитик формулирует в терминах концептуальной модели предметной области, то есть ровно в тех выражениях, в которых он думает о работе своего предприятия.

Витрина данных -- предметно-ориентированная и, как правило, содержащая данные по одному из направлений деятельности компании база данных. Она отвечает тем же требованиям, что и хранилище данных, но в отличие от него, нейтрально к приложениям. В витрине информация храниться оптимизированно с точки зрения решения конкретных задач.

Витрины данных имеют следующие достоинства:
\begin{itemize}
	\item пользователи ведят и работают только с теми данными, которые им действительно нужны,
	
	\item для витрин данных не требуется использовать мощные вычислительные средства.
\end{itemize}

К недостаткам витрин данных можно отнести сложность контроля целостности и противоречивости данных.



\listoffigures\addcontentsline{toc}{section}{Список иллюстраций}

% Источники в "Газовой промышленности" нумеруются по мере упоминания 
\begin{thebibliography}{99}\addcontentsline{toc}{section}{Список литературы}
	\bibitem{lutz:learningpython-2011}{{\emph{Лутц М.} Изучаем Python, 4-е издание. -- Пер. с англ. -- СПб.: Символ-Плюс, 2011. -- 1280~с. }
		
\end{thebibliography}

\end{document}
